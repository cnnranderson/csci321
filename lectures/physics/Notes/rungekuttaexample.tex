\documentclass[twocolumn]{article}
\usepackage[margin=1in]{geometry}
\newcommand{\dt}{\ensuremath{\Delta t}}

\begin{document}

\centerline{\Large Runge Kutta Example}

\centerline{\large Geoffrey Matthews}

\centerline{\today}


Here's a rewrite of the stuff I did in class on Tuesday.
Suppose we have a point moving in space, and the coordinates of the
point are given by $p(t)$.  For a one-dimensional spring, the
coordinates we're interested in are position, velocity, and
acceleration, or 
\[
p(t) = (x(t), v(t), a(t))
\]
For our simple spring, the acceleration is a simple function:
\[
a(t) = -2x(t)
\]
We also presume to know the derivative at any position:
\begin{eqnarray*}
p'(p(t)) &=& p'(x(t), v(t), a(t))\\
         &=& (v(t), a(t), ?)
\end{eqnarray*}
We don't know the derivative of the acceleration, but it will turn out
not to matter.

Now, our system starts at a given time, $t_0$, and advances by $\Delta
t$, giving times $t_n = t_0 + n\Delta t$.
When we talk about positions or derivatives or vectors 
at $n$, it is a simple
translation to actual times:
\begin{eqnarray*}
p(n) &=& p(t_0 + n\Delta t)\\
p'(p(n)) &=& p'(p(t_0 + n\Delta t))
\end{eqnarray*}
For Fourth Order Runge-Kutta integration, we need to calculate four new
vectors, and three new points:
\begin{eqnarray*}
k_1(n) &=& p'(p(n))\Delta t\\
p_2(n) &=& p(n) + \frac{1}{2}k_1(n)\\
k_2(n) &=& p'(p_2(n))\Delta t\\
p_3(n) &=& p(n) + \frac{1}{2}k_2(n)\\
k_3(n) &=& p'(p_3(n))\Delta t\\
p_4(n) &=& p(n) + k_3(n)\\
k_4(n) &=& p'(p_4(n))\Delta t
\end{eqnarray*}
We can now start to fill in our table, with $\Delta t = \frac{1}{2}$
and $a = -2x$.\\
\centerline{\begin{tabular}{c|c|c|c|c}
         &  N &   x &   v &   a   \\\hline
  $p(0)$ &  0 &   8 &   0 & -16   \\
$k_1(0)$ &  0 &   0\dt & -16\dt & ? \\
$k_1(0)$ &  0 &   0 & -8 & ? \\
$p_2(0)$ &  0 &  $8+\frac{1}{2}0$ & $0+\frac{1}{2}(-8)$& ?\\
$p_2(0)$ &  0 &  8 & -4 & -16
\end{tabular}}\\
At this point we can see why we didn't need the derivative of
acceleration.  When we calculate a new point, we get it just from
$a=-2x$.  Continuing:\\
\centerline{\begin{tabular}{c|c|c|c|c}
         &  N &   x &   v &   a   \\\hline
  $p(0)$ &  0 &   8 &   0 & $-16$   \\
$k_1(0)$ &  0 &   0\dt & $-16$\dt & ? \\
$k_1(0)$ &  0 &   0 & $-8$ & ? \\
$p_2(0)$ &  0 &  $8+\frac{1}{2}0$ & $0+\frac{1}{2}(-8)$ & ?\\
$p_2(0)$ &  0 &  8 & $-4$ & $-16$\\
$k_2(0)$ &  0 & $-4$\dt & $-16$\dt & ?\\
$k_2(0)$ &  0 & $-2$ & $-8$ & ?\\
$p_3(0)$ &  0 & $8+\frac{1}{2}(-2)$ & $0+\frac{1}{2}(-8)$ & ?\\
$p_3(0)$ &  0 & 7 & $-4$  & $-14$\\
$k_3(0)$ &  0 & $-4$\dt & $-14$\dt & ? \\
$k_3(0)$ &  0 & $-2$ & $-7$ & ? \\
$p_4(0)$ &  0 & $8+(-2)$ & $0+(-7)$ & ? \\
$p_4(0)$ &  0 & 6 & $-7$ & $-12$\\
$k_4(0)$ &  0 & $-7$\dt & $-12$\dt & ? \\
$k_4(0)$ &  0 & $-\frac{7}{2}$ & $-6$ & ? 
\end{tabular}}\\
Summarizing:\\
\centerline{\begin{tabular}{c|c|c|c|c}
         &  N &   x &   v &   a   \\\hline
  $p(0)$ &  0 &   8 &   0 & $-16$   \\
$k_1(0)$ &  0 &   0 & $-8$ & ? \\
$k_2(0)$ &  0 & $-2$ & $-8$ & ?\\
$k_3(0)$ &  0 & $-2$ & $-7$ & ? \\
$k_4(0)$ &  0 & $-\frac{7}{2}$ & $-6$ & ? 
\end{tabular}}\\
Now we can use the update rule for Runge-Kutta:
\begin{eqnarray*}
p(1) &=& p(0) + \frac{k_1(0)}{6} + \frac{k_2(0)}{3} + \frac{k_3(0)}{3} +
\frac{k_4(0)}{6}\\
x(1) &=& x(0) + \frac06 + \frac{-2}{3} + \frac{-2}{3} +
\frac{7}{(2)(6)}\\
&=& 8 + \frac{-4}{3} + \frac{7}{12}\\
&=& 8 + \frac{-9}{12}\\
&=& 7 + \frac{1}{4}\\
v(1) &=& v(0) + \frac{-8}{6} + \frac{-8}{3} + \frac{-7}{3} +
\frac{-6}{6}\\
&=& \frac{-44}{6}\\
&=& -(7 + \frac{1}{3})\\
a(1) &=& -2(7 + \frac{1}{4})\\
&=& -(14 + \frac{1}{2})
\end{eqnarray*}
So now we have the next row in our table:\\
\centerline{\begin{tabular}{c|c|c|c|c}
         &  N &   x &   v &   a   \\\hline
  $p(0)$ &  0 &   8 &   0 & $-16$   \\
  $p(1)$ &  1 & $7 + \frac{1}{4}$ & $-(7 + \frac{1}{3})$ & $-(14 +
  \frac{1}{2})$
\end{tabular}}\\
and we can start on $p(2)$!
\end{document}


