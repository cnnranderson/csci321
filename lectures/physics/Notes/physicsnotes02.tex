\documentclass[handout,t,compress]{beamer}
\usepackage{etex}
\usetheme{Singapore}

\sloppy
%\usepackage[scaled]{helvet}
%\usepackage{eulervm}

\usepackage{fp-eval}
\usepackage{hyperref}
\usepackage{fancyvrb}
\usepackage{pstricks,pst-node,pst-tree,pst-plot,pst-3dplot,multido}
\usepackage{graphicx}

\usepackage{alltt}


\newcommand{\bframe}[1]{\begin{frame}[fragile]{#1}}

\newcommand{\bbnf}{\begin{center}\begin{tabular}{rcl}}
\newcommand{\bnf}[2]{#1 & ::= & #2 \\}
\newcommand{\ebnf}{\end{tabular}\end{center}}

\newcommand{\myskip}{\vspace{-1em}\hrulefill}
\newcommand{\myrule}[1]{\vspace{-1ex}\centerline{\rule{#1cm}{1pt}}}
\newcommand{\myline}[1]{\centerline{#1}}
\newcommand{\bkt}[1]{\ensuremath{\langle\mbox{#1}\rangle}}
\newcommand{\br}{\mbox{~}|\mbox{~}}

\definecolor{orange}{rgb}{1,.5,0}
\definecolor{pink}{rgb}{1,.75,.75}
\definecolor{ltblue}{rgb}{.75,.75,1}

\newcommand{\be}{\begin{eqnarray*}}
\newcommand{\ee}{\end{eqnarray*}}

\newcommand{\grph}[2]{
\begin{columns}
\column{0.01\textwidth}
\column{0.6\textwidth}
\begin{pspicture}[showgrid=#1](-2,-2)(5,5)
#2
\end{pspicture}
}

\newcommand{\txt}[1]{
\column{0.4\textwidth}
\rput[bl](0,0){\parbox{\textwidth}{
\footnotesize
\begin{itemize}
#1
\end{itemize}}}
\end{columns}}

\newcommand{\myvec}[1]{
\pstThreeDDot[showpoints=true,drawCoor=true](#1)
\pstThreeDLine[arrows=->,linecolor=blue](0,0,0)(#1)
}

\AtBeginSection[]
{
\bframe{Outline}
\tableofcontents[currentsection]
\end{frame}
}

\title{Game Physics Notes 02}
\author{CSCI 321}
\institute{WWU}

\begin{document}\small
\psset{arrowscale=2}

\bframe{~}
\titlepage
\end{frame}

\bframe{Forces}

Newton's second law of motion: $F=ma$

\begin{eqnarray*}
a &=& F/m\\
v' &=& a\\
p' &=& v
\end{eqnarray*}
\begin{quotation}
Corpus omne perseverare in statu suo quiescendi vel movendi
uniformiter in directum, nisi quatenus a viribus impressis cogitur
statum illum mutare. 
\end{quotation}
Or, in English:
\begin{quotation}
Every body perseveres in its state of being at
rest or of moving uniformly straight forward, except insofar as it is
compelled to change its state by force impressed.
\end{quotation}
\end{frame}


\bframe{Forces and Motion}

\begin{eqnarray*}
F &=& ma\\
a &=& F/m\\
v' &=& a\\
p' &=& v
\end{eqnarray*}

\begin{itemize}
\item What we really want to know is: ``How do things move?''\pause
\item If we know the forces and masses, we know the acceleration.\pause
\item If we can integrate the acceleration we can get the velocity.\pause
\item If we can integrate the velocity we can get the position.\pause
\item The problem is integration---generally unsolvable.
\item So we use approximate integration.
\end{itemize}
\end{frame}

\bframe{Euler Integration}

Exact integration would move the point along the blue lines.

\psset{arrowscale=2}
\psset{unit=0.75cm}
\begin{center}
\begin{pspicture}[unit=0.75cm,showgrid=false](-1,-1)(10,6)
%axes
\psline{->}(-1,0)(10,0)
\psline{->}(0,-1)(0,6)

\pscurve[linecolor=blue]{-}(1,1)(5,3)(10,3)
\pscurve[linecolor=blue!50]{-}(1,2)(4.5,3.5)(10,3.5)

%\psline{*->}(1,1)(4.5,3.5)
%\psline{*->}(4.5,3.5)(8.75,4.25)


%\uput[-90](1,1){$\vec{x}_0$}
%\uput[-90](4.5,3.5){$\vec{x}_1$}
%\psdot[dotstyle=BoldMul,dotsize=0.4cm](9,3.1)

%\uput[90](2.75,2.25){$\vec{k}(0)$}
%\uput[90](6.625,3.875){$\vec{k}(1)$}
\end{pspicture}
\end{center}

\end{frame}

\bframe{Euler Integration}

\begin{eqnarray*}
\vec{k}(n) &=& v(t,\vec{x}_n)\Delta t\\
\vec{x}_{n+1} &=& \vec{x}_n + \vec{k}(n)
\end{eqnarray*}

\psset{arrowscale=2}
\psset{unit=0.75cm}
\begin{center}
\begin{pspicture}[unit=0.75cm,showgrid=false](-1,-1)(10,6)
%axes
\psline{->}(-1,0)(10,0)
\psline{->}(0,-1)(0,6)

\pscurve[linecolor=blue]{-}(1,1)(5,3)(10,3)
\pscurve[linecolor=blue!50]{-}(1,2)(4.5,3.5)(10,3.5)

\psline{*->}(1,1)(4.5,3.5)
\psline{*->}(4.5,3.5)(8.75,4.25)


\uput[-90](1,1){$\vec{x}_0$}
\uput[-90](4.5,3.5){$\vec{x}_1$}
\psdot[dotstyle=BoldMul,dotsize=0.4cm](9,3.1)

\uput[90](2.75,2.25){$\vec{k}(0)$}
\uput[90](6.625,3.875){$\vec{k}(1)$}
\end{pspicture}
\end{center}

\end{frame}

\bframe{Euler Integration}

\begin{eqnarray*}
a &=& F/m\\
v' &=& a\\
p' &=& v
\end{eqnarray*}

\bigskip

\begin{centering}
\begin{verbatim}
def update(F, m, dt):
  a = F / m
  v += a * dt
  p += v * dt
\end{verbatim}
\end{centering}

\end{frame}


\bframe{Sample Calculations}
\begin{minipage}{2in}
\begin{eqnarray*}
dt &=& 1 \\
m &=& 10  \\
k &=& 5   \\
f &=& -kx \\
a &=& f/m = -kx/m = -x/2 \\
x' &=& v \\
v' &=& a \\
x_{t+1} &=& x_{t} + x'_{t} = x_{t} + v_{t} \\
v_{t+1} &=& v_{t} + v'_{t} = v_{t} + a_{t} 
\end{eqnarray*}
\end{minipage}
\hfill\begin{minipage}{2in}

Euler:\\
\begin{tabular}{r|rrr|rr}
$t$ & $x$ & $v$ & $a$ \\\hline
0 & 20 & 0 & -10 \\
1 & 20 & -10 & -10 \\
2 & 10 & -20 & -5 \\
3 & -10 & -25 & 5 \\
4 & -35 & -20 & 17 \\
5 & -55 & -3 & 27 \\
\end{tabular}
\end{minipage}

\begin{itemize}
\item
Run {\tt spring.py}
\end{itemize}
\end{frame}


\bframe{Online discussions of Midpoint and Runge Kutta}

Readings:
\begin{itemize}
\item \url{http://www.pixar.com/companyinfo/research/pbm2001/},
  Differential equation basics, and Particle dynamics
\item \url{http://www.nrbook.com/c/}, 16.0, 16.1
\end{itemize}
\end{frame}

\bframe{Midpoint Method}

\begin{eqnarray*}
\vec{k}_1 &=&  v(t, \vec{x}_n)\Delta t\\
\vec{k}_2 &=& v(t+\frac{1}{2}\Delta t, \vec{x}_n+
                     \frac{1}{2}\vec{k}_1)\Delta t \\
\vec{x}_{n+1} &=& \vec{x}_n + \vec{k}_2 
\end{eqnarray*}

\begin{itemize}
\item
Euler method has errors $O(\Delta t^2)$
\item
Midpoint method has errors $O(\Delta t^3)$
\item Can take steps twice as big and get smaller errors:
\begin{eqnarray*}
0.05^2 &=& 0.0025\\
0.10^3 &=& 0.001
\end{eqnarray*}
\end{itemize}
\end{frame}

\bframe{Midpoint Method}
\begin{center}
\begin{pspicture}[unit=0.75cm,showgrid=false](-1,-1)(10,6)
%axes
\psline{->}(-1,0)(10,0)
\psline{->}(0,-1)(0,6)

\pscurve[linecolor=blue]{-}(1,1)(5,3)(10,3)

\psline{*->}(1,1)(8,6)

\rput(0.5,1){$\vec{x}_0$}
\rput(8.5,6){$\vec{k}_1$}
\pscurve{->}(3,5)(5,5)(4,4)(4.5,3.5)
\rput(2,5){\psshadowbox{Halfway point}}

\psdot[dotstyle=BoldMul,dotsize=0.4cm](9,3.1)
\end{pspicture}
\end{center}
\end{frame}

\bframe{Midpoint Method}


\begin{center}
\begin{pspicture}[unit=0.75cm,showgrid=false](-1,-1)(10,6)
%axes
\psline{->}(-1,0)(10,0)
\psline{->}(0,-1)(0,6)

\pscurve[linecolor=blue]{-}(1,1)(5,3)(10,3)
\pscurve[linecolor=blue!50]{-}(1,2)(4.5,3.5)(10,3.5)

\psline{*->}(4.5,3.5)(13,5)

\rput(0.5,1){$\vec{x}_0$}
\psdot[dotsize=.25cm](1,1)

\rput(13,4.5){$\vec{k}_2$}
\pscurve{->}(3,5)(5,5)(4,4)(4.5,3.5)
\rput(2,5){\psshadowbox{Halfway point}}

\psdot[dotstyle=BoldMul,dotsize=0.4cm](9,3.1)
\end{pspicture}
\end{center}
\end{frame}

\bframe{Midpoint Method}


\begin{center}
\begin{pspicture}[unit=0.75cm,showgrid=false](-1,-1)(10,6)
%axes
\psline{->}(-1,0)(10,0)
\psline{->}(0,-1)(0,6)

\pscurve[linecolor=blue]{-}(1,1)(5,3)(10,3)
\pscurve[linecolor=blue!50]{-}(1,2)(4.5,3.5)(10,3.5)

\psline{*->}(4.5,3.5)(13,5)
\psline{*->}(1,1)(9.5,2.5)

\psline[linestyle=dotted]{-}(4.5,3.5)(1,1)
\psline[linestyle=dotted]{-}(13,5)(9.5,2.5)

\rput(0.5,1){$\vec{x}_0$}

\rput(13,4.5){$\vec{k}_2$}
\rput(9.5,1.75){$\vec{k}_2$}
\pscurve{->}(3,5)(5,5)(4,4)(4.5,3.5)
\rput(2,5){\psshadowbox{Halfway point}}

\psdot[dotstyle=BoldMul,dotsize=0.4cm](9,3.1)
\end{pspicture}
\end{center}
\end{frame}

\bframe{Midpoint Method}


\begin{center}
\begin{pspicture}[unit=0.75cm,showgrid=false](-1,-1)(10,6)
%axes
\psline{->}(-1,0)(10,0)
\psline{->}(0,-1)(0,6)

\pscurve[linecolor=blue]{-}(1,1)(5,3)(10,3)

\psline{*->}(1,1)(8,6)
\psline{*->}(1,1)(9.5,2.5)


\rput(0.5,1){$\vec{x}_0$}
\rput(8.5,6){$\vec{k}_1$}
\rput(10,2.5){$\vec{k}_2$}

\psdot[dotstyle=BoldMul,dotsize=0.4cm](9,3.1)

\end{pspicture}
\end{center}
\end{frame}

\bframe{Midpoint Method}

One midpoint method step of size $\Delta t$ 
is more accurate than two Euler steps
of size $\Delta t/2$.

\begin{center}
\begin{pspicture}[unit=0.75cm,showgrid=false](-1,-1)(10,6)
%axes
\psline{->}(-1,0)(10,0)
\psline{->}(0,-1)(0,6)

\pscurve[linecolor=blue]{-}(1,1)(5,3)(10,3)
\pscurve[linecolor=blue!50]{-}(1,2)(4.5,3.5)(10,3.5)

\psline{*->}(1,1)(4.5,3.5)
\psline{*->}(4.5,3.5)(8.75,4.25)
\psline{*->}(1,1)(9.5,2.5)


\rput(0.5,1){$\vec{x}_0$}
\psdot[dotstyle=BoldMul,dotsize=0.4cm](9,3.1)
\end{pspicture}
\end{center}
\end{frame}

\bframe{Sample Calculations}
\begin{minipage}{2in}
\begin{eqnarray*}
dt &=& 1 \\
m &=& 10  \\
k &=& 5   \\
f &=& -kx \\
a &=& f/m = -kx/m = -x/2 \\
x' &=& v \\
v' &=& a \\
x_{t+1} &=& x_{t} + x'_{t} = x_{t} + v_{t} \\
v_{t+1} &=& v_{t} + v'_{t} = v_{t} + a_{t} 
\end{eqnarray*}
\end{minipage}\hfill
\begin{minipage}{2in}
Midpoint:\\
\begin{tabular}{r|rrr|rr}
$t$ & $x$ & $v$ & $a$ \\\hline
0.0 & \rnode{A1}{20} & \rnode{A2}{0} & \rnode{A3}{-10} \\
0.5 & \rnode{B1}{20} & \rnode{B2}{-5} & \rnode{B3}{-10} \\
1.0 & \rnode{C1}{15} & \rnode{C2}{-10} & \rnode{C3}{-7} \\
1.5 & 10 & -13 & -5 \\
2.0 & 2 & -15 & -1 \\
2.5 & -5 & -15 & 2 \\
3.0 & -13 & -13 & 6 \\
3.5 & -19 & -10 & 9 \\
4.0 & -23  & -4 &  11 \\
4.5 & -25 & 1 & 12 \\
5.0 & -24 & 8 & 12\\
\end{tabular}
\nccurve[ncurv=1,angleA=35,angleB=45,arrows=->,arrowscale=2,linecolor=blue]{A1}{B1}
\nccurve[angleA=180,angleB=45,arrows=->,arrowscale=2,linecolor=blue]{A2}{B1}
\nccurve[ncurv=1,angleA=35,angleB=45,arrows=->,arrowscale=2,linecolor=blue]{A2}{B2}
\nccurve[angleA=180,angleB=45,arrows=->,arrowscale=2,linecolor=blue]{A3}{B2}
\end{minipage}

\bigskip \hfill First add half of the derivative.
\end{frame}

\bframe{Sample Calculations}
\begin{minipage}{2in}
\begin{eqnarray*}
dt &=& 1 \\
m &=& 10  \\
k &=& 5   \\
f &=& -kx \\
a &=& f/m = -kx/m = -x/2 \\
x' &=& v \\
v' &=& a \\
x_{t+1} &=& x_{t} + x'_{t} = x_{t} + v_{t} \\
v_{t+1} &=& v_{t} + v'_{t} = v_{t} + a_{t} 
\end{eqnarray*}
\end{minipage}\hfill
\begin{minipage}{2in}
Midpoint:\\
\begin{tabular}{r|rrr|rr}
$t$ & $x$ & $v$ & $a$ \\\hline
0.0 & \rnode{A1}{20} & \rnode{A2}{0} & \rnode{A3}{-10} \\
0.5 & \rnode{B1}{20} & \rnode{B2}{-5} & \rnode{B3}{-10} \\
1.0 & \rnode{C1}{15} & \rnode{C2}{-10} & \rnode{C3}{-7} \\
1.5 & 10 & -13 & -5 \\
2.0 & 2 & -15 & -1 \\
2.5 & -5 & -15 & 2 \\
3.0 & -13 & -13 & 6 \\
3.5 & -19 & -10 & 9 \\
4.0 & -23  & -4 &  11 \\
4.5 & -25 & 1 & 12 \\
5.0 & -24 & 8 & 12\\
\end{tabular}
\nccurve[angleA=0,angleB=45,arrows=->,arrowscale=2,linecolor=blue]{A1}{C1}
\nccurve[angleA=180,angleB=45,arrows=->,arrowscale=2,linecolor=blue]{B2}{C1}
\nccurve[angleA=0,angleB=45,arrows=->,arrowscale=2,linecolor=blue]{A2}{C2}
\nccurve[angleA=180,angleB=45,arrows=->,arrowscale=2,linecolor=blue]{B3}{C2}
\end{minipage}

\bigskip \hfill Then add all the ``half-derivative.''
\end{frame}

\bframe{ Fourth Order Runge-Kutta}
\begin{eqnarray*}
\vec{k}_1 &=& v(t,\vec{x}_n)\Delta t \\
\vec{k}_2 &=& v(t + \frac{1}{2}\Delta t, \vec{x}_n +
\frac{1}{2}\vec{k}_1 )\Delta t \\
\vec{k}_3 &=& v(t+\frac{1}{2}\Delta t,  \vec{x}_n +
\frac{1}{2}\vec{k}_2 )\Delta t \\
\vec{k}_4 &=& v(t+\Delta t,  \vec{x}_n + \vec{k}_3 )\Delta t \\
\vec{x}_{n+1} &=& \vec{x}_n + \frac{\vec{k}_1}{6}
+ \frac{\vec{k}_2}{3}
+ \frac{\vec{k}_3}{3}
+ \frac{\vec{k}_4}{6}
\end{eqnarray*}
\end{frame}

\bframe{Fourth order Runge Kutta}
Tangents calculated at the dots: $\frac{\vec{k}_1}{6}+\frac{\vec{k}_2}{3}+\frac{\vec{k}_3}{3}+\frac{\vec{k}_4}{6}$
\begin{center}
\begin{pspicture}[unit=0.75cm,showgrid=false,arrowscale=2](-1,-1)(10,6)
%axes
\psline{->}(-1,0)(10,0)
\psline{->}(0,-1)(0,6)

\pscurve[linecolor=blue!50]{-}(1,2)(4.5,3.5)(10,3.5)
\pscurve[linecolor=blue]{-}(1,1)(5,3)(10,3)
\pscurve[linecolor=blue!50]{-}(1,-1)(5.25,1.625)(10,2.5)
\pscurve[linecolor=blue!50]{-}(1,3)(5,4.25)(9,4)(10,3.8)

\psline{->}(1,1)(8,6)
\psline{->}(1,1)(9.5,2.25)
\psline{->}(1,1)(9,4)
\psline{->}(1,1)(9.5,.5)

\rput(0.5,1){$\vec{x}_0$}

\rput(8.5,6){$\vec{k}_1$}
\rput(9.9,2.1){$\vec{k}_2$}
\rput(9.5,4.5){$\vec{k}_3$}
\rput(9.9,0.5){$\vec{k}_4$}

\psdot(1,1)
\psdot(4.5,3.5)
\psdot(5.25,1.625)
\psdot(9,4)

\psdot[dotstyle=BoldMul,dotsize=0.4cm](9,3.1)
\end{pspicture}
\end{center}
\end{frame}


\bframe{ Fourth Order Runge-Kutta}
\begin{itemize}
\item
Euler method has errors $O(\Delta t^2)$
\item
Midpoint method has errors $O(\Delta t^3)$
\item
Fourth order Runge Kutta has errors $O(\Delta t^5)$
\item
\begin{eqnarray*}
0.05^2 &=& 0.00250\\
0.10^3 &=& 0.00100\\
0.20^5 &=& 0.00032
\end{eqnarray*}
\end{itemize}

\end{frame}
\bframe{Adaptive stepsize}
\begin{itemize}
\item Change $\Delta t$ as you go along...
\item ...depending on how much things are changing.
\end{itemize}
\end{frame}

\bframe{Differential Equations}
Reading:
\begin{itemize}
\item Strange attractors
\url{http://en.wikipedia.org/wiki/Attractor}
\item The Limits to Growth
\url{http://www.csiro.au/files/files/plje.pdf}
\item Run: {\tt strange??.py}
\end{itemize}
\end{frame}

\bframe{Verlet Integration}

\begin{itemize}
\item A Verlet based approach for 2D game physics (\url{www.gamedev.net})

{\tiny\url{http://www.gamedev.net/page/resources/_/technical/math-and-physics/a-verlet-based-approach-for-2d-game-physics-r2714}}
\item A nice web demo:

{\tiny\url{http://gamedev.tutsplus.com/tutorials/implementation/simulate-fabric-and-ragdolls-with-simple-verlet-integration/}}
\item Can be used as the basis of a collision response system.
\item Run {\tt VerletPhysicsDemo.py}
\item True elastic collisions:
\url{http://en.wikipedia.org/wiki/Elastic_collision}
\item Run {\tt BouncingBalls.py}
\end{itemize}

\end{frame}


\bframe{Symplectic Euler}
\url{http://en.wikipedia.org/wiki/Semi-implicit_Euler_method}
\end{frame}


\bframe{Advanced collision techniques}

Reading:
\begin{itemize}
\item {\tiny
\url{http://www.gamasutra.com/view/feature/3190/advanced_collision_detection_.php}
}
\item Very small objects
\item Fast moving objects
\item Complex objects
\end{itemize}

\end{frame}

\bframe{Many objects}

Reading:
\begin{itemize}
\item Partitioning
\item Sweep and prune
\item \url{http://en.wikipedia.org/wiki/Sweep_and_prune}
\item \url{http://jitter-physics.com/wordpress/?tag=sweep-and-prune}
\item Optional:  nice research on the algorithm:
\url {http://danieljosephtracy.com/Daniel_Joseph_Tracy/Sweep_and_Prune.html}
\end{itemize}

\end{frame}

\bframe{Use a Library}
\begin{itemize}
\item PyMunk: \url{http://code.google.com/p/pymunk/}
\end{itemize}
\end{frame}

\end{document}
