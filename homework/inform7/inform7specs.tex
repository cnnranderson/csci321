\documentclass{article}
\pagestyle{empty}
%\usepackage[margin=1in]{geometry}
\usepackage{hyperref}
\usepackage{graphicx}
\usepackage{multicol}
\setlength{\parindent}{0in}
\setlength{\parskip}{1em}
\begin{document}
\begin{description}
\item[CSCI 321, IF Game Specs, Spring 2014]

\item[Due date: Wednesday, June 3, midnight]
\end{description}


\begin{itemize}
\item 
Minimum game size:  2000 words.  You can see the word count in the
``Errors'' panel after you run the game.  There is no minimum number
of rooms, objects, {\em etc.} as different games have different
requirements. 
\item
Inform7 creates a {\bf folder} with a {\tt .inform} extension.  Zip
this folder and submit to canvas.
\item
Clear out the {\bf Skein} except for one or more ``best'' paths through
your game.  I will play your game briefly, but I want to be able to
advance to any point without spending hours solving your puzzles.
\item
{\bf Bless} your best transcript, as well.  I will read the entire
transcript. 
\item There is no need for separate, printed, documentaiton for the IF
  game.  If there are any special instructions, verbs, {\em etc.},
  they should be printed out, in-game, before the play begins.
\item
Reread the general game specifications for things I'll look for in
your game.  However, since this is interactive fiction, I will be
looking for elements of {\em story} and {\em character}.
\item
Once again, if you want to call my attention to something, perhaps
subtle, that you put into this game, place some autobiographical
information about it in a comment block at the beginning of the game.
[Comments are surrounded by square brackets in Inform7.]
\end{itemize}


\end{document}

