\documentclass{article}
\usepackage[margin=0.75in]{geometry}
\usepackage{alltt}
\usepackage{multicol}
\usepackage{hyperref}
\begin{document}

\centerline{\large \bf Syllabus}

\centerline{\bf CSCI 321, Game Programming, Spring, 2015}

\begin{description}

\item[Instructor:] Geoffrey Matthews, x3797,
 geoffrey dot matthews at wwu dot edu

\item[Office hours:] MTWR 10:00

\item[Lectures:] MTWF 9:00-9:50am, CF 227

\item[Texts and Readings:] \mbox{}
%\begin{multicols}{2}
\begin{itemize}

\item Python and Pygame
\begin{itemize}
 \item \href{http://inventwithpython.com/}
{Invent with Python}
\item \href{http://programarcadegames.com/}
  {Program Arcade Games with Python and Pygame}
\end{itemize}

\item{Artificial Intelligence}
\begin{itemize}
\item \href{http://www.ai-junkie.com/books/toc_pgaibe.html}
{Programming Game AI by Example} (in bookstore)
\end{itemize}

\item{Game Physics}
\begin{itemize}
  \item \href{http://gafferongames.com}{Gaffer on Games}
\item \href{http://graphics.stanford.edu/courses/cs448b-00-winter/papers/phys_model.pdf}
{Physically Based Modeling}
\end{itemize}

\item Blender3D
\begin{itemize}
\item \href{http://www.cdschools.info/blenderbasics/}
  {Blender Basics 4}
\item \href{http://en.wikibooks.org/wiki/Blender_3D:_Noob_to_Pro}
{Blender 3D:  Noob to Pro}
\item \href{http://blendercourse.com/}{Blender Course}
\end{itemize}
\item Other readings as assigned.
\end{itemize}
%\end{multicols}

\item [Webpages:] \mbox{}\begin{itemize}
\item\url{https://wwu.instructure.com} 
\item\url{https://github.com/geofmatthews/csci301}
\end{itemize}


\item
[Content:]
This class is an introduction to the design, programming, ethics,
and business
of computer games.  Topics include:
\begin{multicols}{2}
  \begin{itemize}
  \item
 Game architecture
  \item
 Graphics
  \item
 Physics and simulations
  \item
 Game AI (artificial intelligence)
  \item
 Game Story
 \item Game Mechanics
  \item
 Social aspects of games
  \end{itemize}
\end{multicols}


\item
[Course objectives:]
At the end of this class the student should be able to:
%\begin{multicols}{2}
  \begin{itemize}
  \item Understand basic game architecture:
    \begin{itemize}
    \item Game loop
    \item Event oriented programming
    \end{itemize}
  \item Understand basic nonplayer character AI:
    \begin{itemize}
    \item State oriented behavior
    \item Autonomous motion
    \item A* search
    \item Goal driven behavior
    \item Fuzzy logic decision making
    \end{itemize}
  \item Understand the elements of storytelling and drama:
    \begin{itemize}
    \item Character
    \item Conflict
    \item Plot
    \end{itemize}
  \item Understand blitting and 2D graphics, transparency, animated sprites
  \item Understand the basics of 3D computer graphics
  \item Understand the basics of 3D animation, character modeling,
rigging and skinning
  \item Understand the basics of 2D and 3D physics necessary for games
  \item Understand tradeoffs of 
numerical integration techniques to reduce instabilities and increase speed
  \item Use Python and Pygame to produce a 2D game
  \item Use Blender gamekit to produce a 3D game, including 3D content
  \item Use Inform 7 to produce an interactive fiction game
  \item Understand the basics of the game industry today, job titles,
responsibilities, {\em etc.}
  \item Understand the social and ethical issues of games, including
     game violence, addiction, griefing, and hacking.
  \end{itemize}
%\end{multicols}
\item
[Software:]  We will be using several pieces of free software
in this class:
%\begin{multicols}{2}
\begin{itemize}
  \item GameMaker.  This now comes in two flavors.  Both have free
    versions, but the new one seems to do a poor job importing old
    games.
    \begin{itemize}
    \item\href{http://sandbox.yoyogames.com/}{GameMaker 8.1} The
      traditional one, which I will be using.
    \item\href{http://www.yoygames.com/}{GameMaker Studio 1.1}  A new
      version, with more options for deploying games on phones,
      tablets, {\em etc.}.
    \end{itemize}
  \item
   \url{http://www.python.org}  Python comes in two versions, 2.7 and
   3.3.  I will be using 2.7.
  \item
  \url{http://www.pygame.org}
  \item
  \url{http://numpy.scipy.org}  {\tt numpy} is the package you need.
  \item
  \url{http://www.blender.org}  
  \item
  \url{http://www.inform7.com}
  \end{itemize}
%\end{multicols}

\item [Exams:] All exams will be open book and open notes.  One
  midterm and one final.

\item [Quizzes:] Except for the first and last weeks of class and
  exam days, we
  will have weekly quizzes on Fridays.

\item [Reading:] All students are expected to do the online
  reading assigned throughout the quarter in order to be prepared for
  the weekly quizzes and the exams.

\item [Game Homework:] There will be four game programming
  assignments:
%\begin{multicols}{2}
\begin{itemize}
\item 2D game in GameMaker.
\item 2D game in python and pygame.
\item 3D game in blender gamekit
\item IF (interactive fiction) game in Inform 7
\end{itemize}
%\end{multicols}
Each game must be accompanied by documentation in two forms:
%\begin{multicols}{2}
\begin{itemize}
\item Formatted game manual.
\item In-game documentation.
\end{itemize}
%\end{multicols}

\item [Game Journal:]  All students are required to play at least
  2 hours of games every week.  You must keep a journal on games,
  hours, etc.  More detailed instructions will be handed out in class.

\item [Dream Game:]  All students are required to design a dream
  game.  If you had a team of 10 top programmers and 10 wicked
  artists, what game would you design?   More detailed instructions
  will be handed out in class. 


\item [Grading:] 
%\begin{multicols}{2}
Grades will be based on four games you produce,
  game journal, dream game, weekly quizzes, a midterm, and a final
  exam.  There are no extra credit opportunities for this class.  Put
  all your work into your games and gaming.  Relative weighting of the
  various assessments and assignment of plus and minus is at the
  discretion of the instructor.

$A \ge 90 > B \ge 80 > C \ge 70 > D \ge 60 > F$

\begin{centering}
\begin{tabular}{|rr|}\hline
 Game journal &    10\%\\
 Dream game &      10\%\\
 Quizzes &         10\%\\
 Midterm &         10\%\\
 Final &           20\%\\
 GameMaker game &         10\%\\
 Pygame game &         10\%\\
 Blender3D game & 10\%\\
 IF game &         10\%\\
\hline
\end{tabular}

\end{centering}
%\end{multicols}

\item [Academic dishonesty:] Academic dishonesty policy and
  procedur is discussed in the University Catalog, Appendix D.  All
  students should read this section of the catalog.  It consists of
  consists of misrepresentation by deception or other fraudulent
  means.  In computer science courses this frequently takes the form
  of copying another's program, either a fellow student's program, or
  copying one from the web.  Due diligence should be exercised in the
  labs at all times, since both copying and letting someone else copy
  your program are equally culpable.  Do not walk away from your
  computer in the lab without logging out or locking the screen. 
  Do not share files, even if it is just to ``show them something.''
  Describe it in words, or talk to them in person, never share code.

\item [Collaboration:]  Collaboration with your fellow students is
  a good way to learn.  Feel free to share ideas, solve problems, and
  discuss your programs with other students.  However, collaboration
  is {\em not} copying.  All code should be original.  Remember [    the Simpson's Rule:]  after discussing homework with another
  student, each of you must destroy all written notes, pictures,
  files, {\em etc.} that you shared.  After that, you must watch a
  rerun of {\em the Simpson's}, or do something else unrelated, for
  half an hour.  Then you can take the knowledge you gained from
  another student and put it to work, since it is now not copying, but
  learning.  You have made it your own.

\item [Game contest:] At the end of the quarter we will have a
  game contest, with prizes!  This will be voluntary, and will not
  affect your grade.  Submit your best game
  and all students will be judges.  I (Geoffrey Matthews) will
  tabulate the votes and make a final decision on winners.


\end{description}


\end{document}
